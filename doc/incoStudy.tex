\RequirePackage{lineno}
\documentclass[12pt,draft]{article}
\usepackage[final]{graphicx}
\usepackage{ucs,multicol,natbib,ifdraft,mathtools,xfrac,float,hyperref,array}
\usepackage[utf8x]{inputenc}
\usepackage[T1]{fontenc}
\usepackage[romanian]{babel}
\usepackage[a4paper,margin=2cm]{geometry}
\usepackage[printonlyused,withpage]{acronym}
\usepackage[usenames]{color}
\usepackage[obeyDraft,colorinlistoftodos]{todonotes}

\floatstyle{boxed}
\restylefloat{figure}
\setlength{\columnsep}{25pt}

\graphicspath{ {./img/} }
\DeclareGraphicsExtensions{.pdf,.png,.jpg}

\presetkeys{todonotes}{inline}{}

\bibpunct{(}{)}{;}{a}{,}{,}
 
\makeatletter
\def\closeopenmulticols{%
   \def\@tempa{multicols}%
   \ifx\@tempa\@currenvir
      \end{multicols}%
  \fi 
}

\makeatother
\newcommand\Section[1]{%
 \closeopenmulticols
 \ifdraft{\begin{multicols}{2}[\section{#1}]}{\section{#1}}
}


% close last open multicols
\AtEndDocument{ \closeopenmulticols}

\title{Studiu observațional asupra tratamentului incontinenței urinare de efort la pacientele din ambulator}
\author{Dr.~Andrei~Manu-Marin,~medic~primar~urologie\\Gnosis-EvoMed, str.~Suvenir,~nr.~10,~sect.~2,~București}
\date{\todo{data studiu}}

\begin{document}

\ifdraft{}{\fontsize{6mm}{9mm}\selectfont}
\maketitle\ifdraft{\linenumbers}{ }\todototoc\listoftodos

\begin{abstract}
\ac{IU} este definită ca orice pierdere involuntară a urinei. \ac{IU} face parte din categoria de simptome ale tractului urinar inferior (prescurtare: \ac{LUTS}) care includ dificultăți atât legate de stocarea urinei cât și de eliminarea ei, \ac{IU} fiind în categoria simptome de
stocare. \ac{IU} poate fi caracterizată în plus prin datele obținute în urma anamnezei și a contextului simptomelor descrise de pacient. \todo{Mai multe detalii despre studiu}
\end{abstract}
  
\Section{Introducere}
\ac{IU} este definită ca orice pierdere involuntară a urinei. \ac{IU} face parte din categoria de simptome ale tractului urinar inferior (prescurtat, \ac{LUTS}) care includ dificultăți atât legate de stocarea urinei cât și de eliminarea ei, \ac{IU} fiind în categoria simptome de stocare. \ac{IU} poate fi caracterizată în plus prin datele obținute în urma anamnezei și a contextului simptomelor descrise de pacient.  

\ac{IUI} se definește ca pierderea de urină precedată de senzația intensă de a urina, numită imperiozitate. \ac{IUE} se definește ca eliminarea involuntară de urină asociată cu anumite activități fizice (de ex. strănut și tuse). \ac{IUM} include caracteristici atât ale \ac{IUI} cât și ale \ac{IUE}. \todo{Informații despre cercetare anterioara}

%%%%    
\Section{Metode}

\subsection{Protocolul clinic}
  Studiul este unul observațional care evaluează răspunsul unui grup de pacienți tratat ambulatoriu pe o perioada de 12 săptămâni de tratament. Au fost înrolați 50 pacienți de ambele sexe(F=31,M=19) pe o perioada de 8 săptămâni ($\pm$ 4 săptămâni). Criteriile de includere au fost:
  \begin{itemize}
    \item Incontinență urinară timp de cel puțin trei luni
    \item Bărbați și femei adulți tratați în ambulator
    \item Mai mult de 1 episod de IU pe zi conform jurnalului micțiunilor de 2 zile
    \item IU dovedită în timpul testelor urodinamice
   \end{itemize}
 
 Criteriile de excludere au fost:
  \begin{itemize}
    \item Pierdere continuă de urină.
    \item Sarcină sau planificare a unei sarcini în interval de 1 an.
    \item Infecție activă a tractului urinar.
    \item Retenție urinară.
    \item Antecedente de tumori ale vezicii urinare, intervenție chirurgicală împotriva cancerului la nivel pelvin (amputație de rect, histerectomie radicală)
    \item Iradiere pelvină
    \item Sub medicație curentă pentru incontinență.
    \item Condiție neurologică care afectează funcția vezicii urinare.
    \item Deficiență mintală 
    \item Intervenție chirurgicală anterioară pentru IU
    \item Intervenție chirurgicală anterioară pentru patologia prostatei 
   \end{itemize}
  Pacienții incluși au efectuat proceduri de recuperare și stimulare periferică timp de 8 săptămâni constând în 3 sesiuni de \ac{SEP} pe săptămâna pentru 8 săptămâni și 3 sesiuni de fizioterapie pe săptămâna pentru 4 săptămâni începând din săptămâna 5. Ulterior, pacienții au fost instruiți sa facă exerciții fizice acasă, fără supraveghere timp de 4 săptămâni. O vizita de evaluare și urmărire a fost efectuata la 6 luni de la includerea în studiu.
  
  Pacienților le-au fost administrate la începutul și sfârșitul tratamentului 4 chestionare care cuprind evaluări subiective folosind o scală psihometrica Likert:
  \begin{itemize}
    \item \ac{CEII} -- sunt enumerate 7 activități uzuale și se cere pacienților sa evalueze pe o scara discreta de la 0 la 3 (valori mai mari indică impact negativ mai important), care este impactul pierderilor de urină. Este înregistrată suma evaluărilor.
    \item \ac{CVDSU} -- evaluează pe o scara discreta de la 0 la 7, impresia asupra calității vieții viitoare condiționata de prezenta pierderilor de urină. Valori mai mari reprezintă o calitate a vieții inferioara.
    \item \ac{VAS} -- evaluează pe o scara discreta de la 0 la 10, impresia asupra calității vieții actuale condiționată de prezenta pierderilor de urină. Valori mai mari reprezintă o calitate a vieții inferioară.
    \item \ac{IGPI} -- evaluează pe o scara discreta de la 0 la 7, impresia pacienților asupra efectului tratamentului. 1 reprezintă efect pozitiv maxim, 4 reprezintă nici un efect, 7 reprezintă efect negativ maxim.
  \end{itemize}
  De asemenea, următorii parametrii obiectivi au fost înregistrați folosind chestionare administrate la începutul și sfârșitul tratamentului pentru a putea urmării eficacitatea acestuia:
  \begin{itemize}
    \item I2D -- înregistrează numărul de episoade de incontinent din ultimele 2 zile premergătoare completării chestionarului. 
    \item \ac{FEFMP} -- înregistrează calitatea contracției musculaturii pelvine pe o scara discreta de la 1 la 5 cu valori mai mari reprezentând o contracție puternică. 
    \item \ac{USS} -- înregistrează numărul de vizite la medicul de familie și medicul specialist urolog/ginecolog în ultimele 3 luni anterioare administrării chestionarului, legate de prezenta pierderilor de urină.
  \end{itemize}

\subsection{Metode statistice}
  Pentru a analiza datele au fost folosite mai multe metode matematice bazate atât pe abordarea asa zis fregventionista cât și pe cea bayesiana. Datele au fost analizate folosind mediul de dezvoltare numit R (http://www.r-project.org/). Mai jos sunt prezentate pe scurt câteva dintre metode împreuna cu referințe bibliografice pentru mai multe detalii.

\subsubsection{Testul Wilcoxon}
 \textbf{Testul Wilcoxon} este un test non-parametric pentru a testa ipoteza statistica de egalitate a primului moment pentru doua populații care se folosește atunci când distribuita celor 2 populații nu este normala (alternativa pentru populații normale este Testul Student t, sau Testul Z). 
 Populațiile trebuie sa îndeplinească următoarele condiții:
 \begin{itemize}
  \item Datele examinate provin din aceeași populație
  \item Datele sunt aleatoare, independente și identic distribuite
  \item Datele sunt reprezentate prin numere întregi sau reale
  \item Distribuția este simetrică în jurul valorii medianei.
 \end{itemize}
  Testul împerechează datele din cele 2 populații $(x_{2,i},x_{1,i})$, elimina perechile de valori identice, și le sortează în ordinea crescătoare a diferenței absolute $|x_{2,i}-x_{1,i}|$ cu $R_i=1, ..., N_r$ semnificând rangul perechii $(x_{2,i},x_{1,i})$ după ordonare. 
  Ulterior se calculează statistica $W = |\sum_{i=1}^{N_r} [sgn(x_{2,i} - x_{1,i}) \cdot R_i]| $ și un scor $p = \frac{W - 0.5}{\sigma_W}, \sigma_W = \sqrt{\frac{N_r(N_r + 1)(2N_r + 1)}{6}}$. 
  Dacă scorul este mai mare decât un prag convențional ales $0.05$ atunci ipoteza $H_0$ de egalitate a primului moment este rejectata. 
  Pentru detalii vezi \citep{wilcoxon45,siegel56}.

\subsubsection{Testul Kolmogorov–Smirnov}
 \textbf{Testul Kolmogorov–Smirnov} este un test non-parametric pentru ipoteza statistică de proveniență din aceeași distribuție continuă și unidimensională pentru doua eșantioane care se folosește atunci când distribuția nu este normală (teste mai puternice pentru a determina normalitatea datelor sunt  Shapiro–Wilk sau Anderson–Darling \citep{Stephens74} ). 
 Plecând de la distribuția empirică descrisa de funcția $F_n(x)={1 \over n}\sum_{i=1}^n I_{X_i\leq x}$ unde $X_i$ sunt variabile independente și identic distribuite iar $I_{X_i\leq x}$ este funcția indicator egala cu $1$ dacă $X_i\leq x$ și cu $0$ în rest, se calculează statistica Kolmogorov–Smirnov $D_{n,n'}=\sup_x |F_{1,n}(x)-F_{2,n'}(x)|$ pentru o fiecare distribuție empirică $F_{i,n}(x)$ data. 
 Teorema lui Kolmogorov arata că ipoteza nula este rejectata cu o probabilitate p dacă $D_{n,n'}\sqrt{\frac{n n'}{n + n'}}>K_\alpha$ unde $K_\alpha$ este obținut din $Pr(K\leq K_\alpha)=1-\alpha$ cu $Pr(K\leq x)$ fiind distribuția cumulativa de probabilitate data de $Pr(K\leq x)=1-2\sum_{k=1}^\infty (-1)^{k-1} e^{-2k^2 x^2}=\frac{\sqrt{2\pi}}{x}\sum_{k=1}^\infty e^{-(2k-1)^2\pi^2/(8x^2)}$. 
 Pentru detalii vezi \citep{stuart99}.

\subsubsection{Testul Student t}
 \textbf{Testul Student t} sau \textbf{testul t} este un test parametric pentru ipoteza statistică nula de egalitate a mediei intre 2 eșantioane ($X_1,X_2$) sau intre media unui eșantion și o valoare specificata. Statistica testata este $t = \frac{\bar {X}_1 - \bar{X}_2}{S_{X_1X_2} \cdot \sqrt{\frac{1}{n_1}+\frac{1}{n_2}}}$ cu $ S_{X_1X_2} = \sqrt{\frac{(n_1-1)S_{X_1}^2+(n_2-1)S_{X_2}^2}{n_1+n_2-2}}.$ și $S_{X_1},S_{X_1}$ sunt deviațiile standard iar $\bar {X}_1,\bar{X}_2$ sunt mediile ale eșantioanelor $X_1,X_2$.
 Eșantioanele trebuie sa îndeplinească următoarele condiții:
 \begin{itemize}
  \item Provin din aceeași populație cu o distribuție normala
  \item Datele sunt aleatoare, independente și identic distribuite
  \item Deviația standard $S^2$ a eșantioanelor are o distribuție de tipul $\chi^2$ (chi pătrat)
 \end{itemize}
 Testul t este robust la variațiile datelor de la normalitate dar se vor urmării câteva recomandări înainte de aplicarea lui:
 \begin{itemize}
  \item Sa se verifice folosind metoda grafica dacă datele urmăresc o distribuție de tip ``cocoașă``
  \item Dacă dispersia $var(x)$ celor 2 eșantioane nu este egala (testabila folosind testul F, Levene, Bartlett sau cu un grafic Q-Q) trebuie aplicata corecția Welch care modifica statistica t în $t = {\overline{X}_1 - \overline{X}_2 \over s_{\overline{X}_1 - \overline{X}_2}}$ cu $s_{\overline{X}_1 - \overline{X}_2} = \sqrt{{s_1^2 \over n_1} + {s_2^2  \over n_2}}$
  \item Comparat cu testul Wilcoxon, testul t este potrivit pentru analiza datelor colectate folosind scale Likert deoarece are rezultate comparabile cu acesta în cazurile uzuale și chiar superioare dacă premizele testului Wilcoxon nu sunt îndeplinite: distribuția este multi-modala sau puternic deplasata spre extreme. Vezi \citep{clason1994analyzing,deWinter12}.
 \end{itemize}
 Pentru detalii vezi \citep{welch1947generalisation}.
  
%%%%    
\Section{Rezultate}
\subsection{Populația}
  Un număr de 50 de pacienți au fost observați. Dintre aceștia 62\% (N=31) sunt de sex feminin iar 38\% (N=19) sunt de sex masculin (proporția sexelor în grupa populației urbane cu vârste cuprinse intre 27 și 83 ani la nivel național conform \citep{insee2011} este de 47\% M și 53\% F). 25 dintre aceștia suferă de \ac{IUE} și 25 de \ac{IUI}.
  Vârsta pacienților de sex feminin este distribuita normal în jurul mediei de 50 de ani și 7 luni ($\sigma=14.3,min=27,max=77$) iar cea a pacienților de sex masculin este o combinație de distribuții normale centrate în jurul mediilor de 46 respectiv 75 ani ($\sigma_{1}=12.3 , \sigma_{2}=9.2,min=30,max=83$).
  Pentru a evalua reprezentativitatea eșantionului relativ la distribuția vârstelor în cadrul populației din Romania am apelat la datele oficiale din \citep{insee2011} care detaliază numărul de cetățeni romani pe sexe și categorie urban/rural pentru fiecare vârstă la data de 1 iulie 2010. 
  Analiza statistică s-a efectuat folosind testul Wilcoxon iar concluzia este că atât eșantionul de sex feminin ($p=0.9964$) cât și cel de sex masculin($p=0.9967$) corespund cu distribuția generala în populația urbana a României.
  %
  \begin{figure}[H]
   \centering
   \includegraphics[width=0.8\linewidth]{incoVarstaSex}
   \caption{Distribuția sexelor participanților la studiu}
   \label{fig:Distributia sexelor participantilor la studiu}
  \end{figure}
  %
  Din punct de vedere al greutății am evaluat indicatorul \ac{BMI} conform cu pragurile recomandate de \citep{whobmi06}. 
  Astfel, pentru pacienții de sex feminin avem 13 persoane cu greutate normala ($BMI<25.0$, NOR), 16 supraponderale ($25.0 \geq BMI <30.0$, OVR) și 2 obeze($BMI \geq 30.0$, OBE). 
  Pentru pentru pacienții de sex masculin avem 3 persoane cu greutate normala, 12 supraponderale și 4 obeze.   
  %
  \begin{table}[H]
   \centering
   \begin{tabular}{ |l|l|l|l| }
    \hline
    Sex & NOR & OVR & OBE \\ \hline
    F & 13 & 16 & 2 \\ \hline
    M & 3 &  12 & 4 \\ \hline
   \end{tabular}
   \caption{Numărul de persoane din fiecare categorie \ac{BMI} pe sexe}
   \label{tab:BMIgSex}
  \end{table}
  %
  \begin{figure}[H]
    \centering
    \includegraphics[width=0.8\linewidth]{incobmiDens}
    \caption{Distribuția \ac{BMI} pe sexe. Zona mai deschisa marchează persoanele supraponderale și cea mai închisă pe cele obeze}
    \label{fig:incobmiDens}
  \end{figure}
  %
  Distribuția \ac{BMI} pe grupa de vârstă și pe sexe a fost evaluată la nivel național conform \citep{EHIS09}, care oferă informații detaliate despre incidenta problemelor de nutriție în rândul tarilor membre ale Uniunii Europene. 
  Din cauza eșantionului foarte mic, nu se poate trage concluzia că populația studiată provine dintr-un eșantion aleator la nivel național dar examinând graficul din Figura~\ref{fig:incoBMIvsEHIS-OOB} se poate observa (cu excepția unor situații particulare - de exemplu toate persoanele de sex masculin din grupa de vârstă 25-44 ani sunt supraponderale sau obeze) că valorile procentelor urmăresc distribuția națională.
  Pentru a testa dacă eșantioanele provin din aceeași distribuție comună am folosit testul \ac{KS} care a dat o probabilitate de 60\% pentru persoanele de sex feminin și de doar 12.4\% pentru persoanele de sex masculin indicând că datele nu sunt suficiente pentru a susține în mod concludent reprezentativitatea eșantionului sau că există un bias de selecție a pacienților în funcție de \ac{BMI}.
  %
  \begin{figure}[H]
    \centering
    \includegraphics[width=0.8\linewidth]{incoBMIvsEHIS-OOB}
    \caption{Distribuția procentului de persoane obeze în populația studiata (EIU) și în populația generala (EHIS) }
    \label{fig:incoBMIvsEHIS-OOB}
  \end{figure}
  %
  \closeopenmulticols\begin{table}[H]
    \centering
    \begin{tabular}{ |l|l|l|l|l| }
      \hline
      Grupa de vârstă & Sex & Categorie BMI & Număr persoane & Procent \\ \hline
      25-44 & F & NOR & 7  & 63.6 \\ \hline
      25-44 & F & OVR & 3  & 27.3 \\ \hline
      25-44 & F & OBE & 1  &  9.1 \\ \hline
      25-44 & M & NOR & 0  &  0.0 \\ \hline
      25-44 & M & OVR & 4  & 80.0 \\ \hline
      25-44 & M & OBE & 1  & 20.0 \\ \hline
      45-64 & F & NOR & 3  & 21.4 \\ \hline
      45-64 & F & OVR & 10 & 71.4 \\ \hline
      45-64 & F & OBE & 1  &  7.1 \\ \hline
      45-64 & M & NOR & 1  & 20.0 \\ \hline
      45-64 & M & OVR & 3  & 60.0 \\ \hline
      45-64 & M & OBE & 1  & 20.0 \\ \hline
      65-74 & F & NOR & 3  & 50.0 \\ \hline
      65-74 & F & OVR & 3  & 50.0 \\ \hline
      65-74 & F & OBE & 0  &  0.0 \\ \hline
      65-74 & M & NOR & 2  & 22.2 \\ \hline
      65-74 & M & OVR & 5  & 55.6 \\ \hline
      65-74 & M & OBE & 2  & 22.2 \\ \hline
    \end{tabular}
    \caption{Numărul de persoane și procentul din totalul de persoane dintr-o grupa de vârstă din fiecare categorie \ac{BMI} pe sexe și pe grupa de vârstă}
    \label{tab:bmigCounts}
  \end{table}
  %
  
  \ifdraft{\begin{multicols}{2}}{}
  Dintre persoanele de sex feminin ($N=31$), 17 sunt la menopauza, 2 paciente au înregistrate câte 3 nașteri, 10 paciente au câte 2 nașteri, 13 paciente au câte o naștere și 6 paciente nu au nici o naștere. Pentru a compara fertilitatea eșantionului cu media naționala am calculat indicatorul \ac{ICF} după definiția folosita în \citep{insee2011} care a rezultat egal cu $1.125$ fata de media naționala pe anul 2010 de $1.3$ iar rezultatele sub forma grafica sunt afișate în Figura~\ref{fig:incoNasteriICF}.
  %
  \begin{figure}[H]
    \centering
    \includegraphics[width=0.8\linewidth]{incoNasteriICF}
    \caption{Variația ICF cu vârsta pacienților. Se observa convergenta asimptotica către statistica naționala (linia orizontala roșie) pe măsura ce sunt incluse persoanele trecute de perioada fertila }
    \label{fig:incoNasteriICF}
  \end{figure}
  %
  
  Studiul a înregistrat și  date referitor la co-morbiditatea pacienților colectând date despre prezenta următoarelor condiții medicale: bronșita cronica, diabet, sindrom Parkinson, mieilita, spina bifida, depresie, fractura vertebrala, fractura de coloana sau \ac{AVC}. 24 de pacienți nu au raportat nici o condiție. Sumarul datelor este prezentat în tabelul \ref{tab:comoSumary}. 
  %
  \begin{table}[H]
    \centering
    \begin{tabular}{ |l| >{\centering\arraybackslash}p{1.4cm} | }
      \hline
      Condiție medicala & Număr \newline pacienți \\ \hline
      AVC & 7 \\ \hline
      DEPRESIE & 3 \\ \hline
      DIABET & 6 \\ \hline
      FRACTURA COLOANA & 2 \\ \hline
      FRACTURA VERTEBRALA & 1 \\ \hline
      MIELITA & 3 \\ \hline
      PARKINSON & 3 \\ \hline
      SPINA BIFIDA & 1 \\ \hline
    \end{tabular}
    \caption{Condiția medicala și numărul de persoane pentru fiecare}
    \label{tab:comoSumary}
  \end{table}
  %
  După cum se observa în Figura \ref{fig:incoComoCntBySex}, distribuția condițiilor medicale variază foarte mult în funcție de sexul pacientului astfel încât pacienții de sex masculin raportează cele mai multe cazuri de co-morbiditate ($N_B=17$ vs $N_F=9$) chiar dacă numărul lor total este mai mic în eșantion ($Total_B=19$ vs $Total_F=31$).
  \begin{figure}[H]
    \centering
    \includegraphics[width=0.8\linewidth]{incoComoCntBySex}
    \caption{Numărul de condiții medicale pentru fiecare sex. }
    \label{fig:incoComoCntBySex}
  \end{figure}
  %

\subsection{Efecte}
  Analiza datelor raportate de pacienți (atât cele subiective cât și cele obiective) a arătat o îmbunatățire consistenta a tuturor valorilor măsurate. Pentru rigurozitate am folosit testul t pentru a rejecta ipoteza nula conform căreia nu exista nici o diferența după aplicarea tratamentului în parametrii măsurați. La toți parametrii, probabilitatea ca ipoteza nula sa fie adevărată este $\ll 0.05$ ceea ce înseamnă ca efectul este real din punct de vedere statistic. Un sumar al parametrilor împreună cu intervale de încredere estimate de testul t este prezentat în tabela~\ref{tab:resTestSummary}.
  \begin{table}[H]
    \centering
    \begin{tabular}{|l|c|c|}
      \hline
       & $Pr(>|t|)$ & 95 \% CI \\   \hline
      I2D & 		$1.8e-22$ 	& $[ 6.50, 8.22 ]$ \\ \hline
      CEII & 		$4.4e-32$ 	& $[ 10.83, 12.49 ]$ \\ \hline
      CVDSU & $8e-30$ 		& $[ 3.96, 4.64 ]$ \\ \hline
      VAS & 		$1.8e-24$ 	& $[ 5.50, 6.78 ]$ \\ \hline
      USS & 		$7.2e-11$ 	& $[ 1.27, 2.09 ]$ \\ \hline
      FEFMP & $2.1e-25$ 	& $[ -2.31, -1.89 ]$ \\ \hline
    \end{tabular}
    \caption{Rezultatele testului t pentru parametrii măsurați} 
    \label{tab:resTestSummary}
  \end{table}
  %
  \begin{figure}[H]
    \centering
    \includegraphics[width=0.8\linewidth]{incoResVariousStack}
    \caption{\ac{CEII},\ac{CVDSU},\ac{VAS} înainte și după tratament}
    \label{fig:incoResVariousStack}
  \end{figure}
  %
  În figura~\ref{fig:incoResVariousStack} se observa cum toți parametrii au migrat către valori considerate pozitive, aici reprezentate prin nuanțe de verde. 
  
  Un alt parametru care a înregistrat o îmbunatățire este \acf{USS}, care după cum se vede în figura~\ref{fig:incoResUSS} indica o scădere cu 71\% în agregat a numărului de prezentări la medic cauzate de probleme de incontinenta.
  %
  \begin{figure}[H]
    \centering
    \includegraphics[width=0.8\linewidth]{incoResUSS}
    \caption{\acf{USS}}
    \label{fig:incoResUSS}
  \end{figure}
  %
  
  Datele obiective (numărul de episoade de incontinenta pe 2 zile și \ac{FEFMP}) arata o îmbunatățire în urma tratamentului conform tabelului~\ref{tab:resTestSummary}. Pentru a evalua efectul tratamentului asupra \ac{FEFMP} am folosit modele lineare cu efecte fixe si testul ANOVA. Modelul selectat ca fiind cel mai bun folosind ANOVA este $y_{it} = X_{it}\mathbf{\beta}+\alpha_{i}+u_{it}$ unde $y_{it}$ este valoarea FEFMP pentru individul $i$ la momentul $t \in [PRE,POST]$ iar $X_{it}$ este vectorul de regresie $\left(\!\begin{array}{c}Trt\\group\end{array}\right)$. Dupa cum se vede din tabela~\ref{tab:resFEFMPlm}, tratamentul este foarte semnificativ iar un grad mare de semnificatie il are si cauza incontinentei urinare, grupul care sufera de \ac{IUI} avand un raspuns mai prost la tratament fata de cei ce sufera de \ac{IUE} dar fata de efectul tratamentului, influenta cauzei este de 5 ori mai slabă.
  Pentru I2D, am inclus in modelul linear si un termen legat de numarul de nasteri dar rezultatele nu indica semificatie statistica nici pentru cauza sindromului si nici pentru numarul de nasteri. Mai precis, numarul de nasteri este corelat slab ($p=0.17$ insuficient pentru pragul de  relevanta statistica ales de $p<0.05$) cu datele conform tabelului~ \ref{tab:resI2Dlm}.
  \begin{table}[H]
		\centering
		\begin{tabular}{rrrrr}
			\hline
			& Est. & $\sigma~~~$ & Pr($>|t|$) \\ \hline
			(Intercept) & 2.21 & 14.35 & 0.000 \\ 
			TrtPOST & 2.10  & 11.81 & 0.000 \\ 
			groupIUI & -0.38 & -2.14 & 0.035 \\ \hline
		\end{tabular}
		\caption{Rezultatele modelului linear pentru FEFMP} 
		\label{tab:resFEFMPlm}
	\end{table}
  
	\begin{table}[H]
		\centering
		\begin{tabular}{rrrrr}
			\hline
			& Est. & $\sigma~~~$ & Pr($>|t|$) \\ \hline
			(Intercept) & 7.99 & 0.71 & 0.00 \\ 
			TrtPOST & -7.06 & 0.69 & 0.00 \\ 
			Nasteri & 0.57 & 0.41 & 0.17 \\
			groupIUI & 0.67 & 0.88 & 0.44 \\ \hline
		\end{tabular}
		\caption{Rezultatele modelului linear pentru I2D} 
		\label{tab:resI2Dlm}
	\end{table}
  Impresiile pacienților despre efectele tratamentului, colectate la sfârșitul studiului clinic coincid cu rezultatele noastre, mai mult de 70\% ($n=39$) raportând ca se simt mai bine sau mult mai bine fata de situația anterioara.
  \begin{figure}[H]
    \centering
    \includegraphics[width=0.8\linewidth]{incoResIGPI}
    \caption{\acf{IGPI} la sfârșitul tratamentului}
    \label{fig:incoResIGPI}
  \end{figure}
  %
  
 \Section{Concluzii}
  
   
  \closeopenmulticols
  
  \bibliographystyle{plainnat}\bibliography{incoStudy}
  
  \section*{Glosar}
  \begin{acronym}[LUTS]
    \acro{IU}{Incontinența Urinară}
    \acro{IUI}{Incontinența Urinară prin Imperiozitate}
    \acro{IUE}{Incontinența Urinară de Efort}
    \acro{IUM}{Incontinența Urinară Mixtă}
    \acro{LUTS}{Lower Urinary Tract Symptoms}
    \acro{SEP}{Stimulare Electrica Periferica}
    \acro{BMI}{Body-Mass Index}
    \acro{KS}{Kolmogorov–Smirnov}
    \acro{ICF}{Indicatorul Conjunctural de Fertilitate}
    \acro{AVC}{Accident vascular cerebral}
    \acro{CEII}{Chestionar de Evaluare a Impactului Incontinentei}
    \acro{CVDSU}{Calitatea Vieții Datorata Simptomelor Urinare}
    \acro{VAS}{Scala Vizual Analogică pentru evaluarea gradului de îmbunătățire a calității vieții}
    \acro{FEFMP}{Fisa de Evaluare a Forței Musculaturii Perineale}
    \acro{IGPI}{Impresia Globala a Pacientului de Îmbunătățire}
    \acro{USS}{Utilizarea Serviciilor De Sănătate}
  \end{acronym}
 
 \listoffigures
 \listoftables
   
\end{document}
