\RequirePackage{lineno}
\documentclass[11pt,draft]{article}
\usepackage[final]{graphicx}
\usepackage{ucs,multicol,natbib,ifdraft,mathtools,xfrac,float,hyperref}
\usepackage[utf8x]{inputenc}
\usepackage[T1]{fontenc}
\usepackage[romanian]{babel}
\usepackage[a4paper,margin=2cm]{geometry}
\usepackage[printonlyused,withpage]{acronym}
\usepackage[usenames]{color}
\usepackage[obeyDraft,colorinlistoftodos]{todonotes}

\floatstyle{boxed}
\restylefloat{figure}

\graphicspath{ {./img/} }
\DeclareGraphicsExtensions{.pdf,.png,.jpg}

\presetkeys{todonotes}{inline}{}

\bibpunct{(}{)}{;}{a}{,}{,}
 
\makeatletter
\def\closeopenmulticols{%
   \def\@tempa{multicols}%
   \ifx\@tempa\@currenvir
      \end{multicols}%
  \fi 
}
\makeatother
\newcommand\Section[1]{%
 \closeopenmulticols
  \begin{multicols}{2}[\section{#1}]}


% close last open multicols
\AtEndDocument{ \closeopenmulticols}

\title{Studiu observațional asupra tratamentului incontinenței urinare de efort la pacientele din ambulator}
\author{Dr.~Andrei~Manu-Marin,~medic~primar~urologie\\Gnosis-EvoMed, str.~Suvenir,~nr.~10,~sect.~2,~București}
\date{}

\begin{document}

  \setlength{\columnsep}{25pt}
  \maketitle
  \ifdraft{ \linenumbers}{ }
  
 \todototoc
 \listoftodos

    \begin{abstract}
	\ac{IU} este definită ca orice pierdere involuntară a urinei. \ac{IU} face parte din categoria de simptome ale tractului urinar inferior (prescurtare: \ac{LUTS}) care includ dificultăți atât legate de stocarea urinei cât și de eliminarea ei, \ac{IU} fiind în categoria simptome de stocare. \ac{IU} poate fi caracterizată în plus prin datele obținute în urma anamnezei și a contextului simptomelor descrise de pacient. \todo{Mai multe detalii despre studiu}
  \end{abstract}
  
      \Section{Introducere}
	\ac{IU} este definită ca orice pierdere involuntară a urinei. \ac{IU} face parte din categoria de simptome ale tractului urinar inferior (prescurtat, \ac{LUTS}) care includ dificultăți atât legate de stocarea urinei cât și de eliminarea ei, \ac{IU} fiind în categoria simptome de stocare. \ac{IU} poate fi caracterizată în plus prin datele obținute în urma anamnezei și a contextului simptomelor descrise de pacient.  

	\ac{IUI} se definește ca pierderea de urină precedată de senzația intensă de a urina, numită imperiozitate. \ac{IUE} se definește ca eliminarea involuntară de urină asociată cu anumite activități fizice (de ex. strănut și tuse). \ac{IUM} include caracteristici atât ale \ac{IUI} cât și ale \ac{IUE}. \todo{Informații despre cercetare anterioara}
    
     \Section{Metode}
      \subsection{Protocolul clinic}
	Studiul este unul observațional care evaluează răspunsul unui grup de pacienți tratat ambulatoriu pe o perioada de 12 săptămâni de tratament. Au fost înrolați 50 pacienți de ambele sexe(F=31,M=19) pe o perioada de 2 luni ($\pm$ 1 luna). Pacienții au efectuat proceduri de recuperare și stimulare periferica timp de 8 săptămâni constând în 3 sesiuni de \ac{SEP} pe săptămâna pentru 8 săptămâni și 3 sesiuni de fizioterapie pe săptămâna pentru 4 săptămâni începând din săptămâna 5. Ulterior, pacienții au fost instruiți sa facă exerciții fizice acasă, fără supraveghere timp de 4 săptămâni. O vizita de evaluare și urmărire a fost efectuata la 6 luni de la includerea în studiu.
      \subsection{Metode statistice}
	Pentru a analiza datele au fost folosite mai multe metode matematice bazate atât pe abordarea asa zis fregventionista cât și cea bayesiana. Datele au fost analizate folosind mediul de dezvoltare numit R (http://www.r-project.org/). Mai jos sunt prezentate pe scurt câteva dintre metode împreuna cu referințe bibliografice pentru mai multe detalii.
	\subsubsection{Testul Wilcoxon}
	Testul Wilcoxon este un test non-parametric pentru a testa ipoteza statistica de inegalitate a primului moment pentru doua populații care se folosește atunci când distribuita celor 2 populații nu este normala (alternativa pentru populații normale este Testul Student t, sau Testul Z). Populațiile trebuie sa îndeplinească următoarele condiții:
	 \begin{itemize}
	 \item Datele examinate provin din aceeași populație
	 \item Datele sunt randomizate și independente
	 \item Datele sunt reprezentate prin numere întregi sau reale
	 \item Distribuția este simetrica în jurul valorii medianei.
	 \end{itemize}
	Testul împerechează datele din cele 2 populații $(x_{2,i},x_{1,i})$, elimina perechile de valori identice, și le sortează în ordinea crescătoare a diferenței absolute $|x_{2,i}-x_{1,i}|$ cu $R_i=1, ..., N_r$ semnificând rangul perechii $(x_{2,i},x_{1,i})$ după ordonare. Ulterior se calculează statistica $W = |\sum_{i=1}^{N_r} [sgn(x_{2,i} - x_{1,i}) \cdot R_i]| $ și un scor $p = \frac{W - 0.5}{\sigma_W}, \sigma_W = \sqrt{\frac{N_r(N_r + 1)(2N_r + 1)}{6}}$. Dacă scorul este mai mare decât un prag convențional ales $0.05$ atunci ipoteza $H_0$ de egalitate a primului moment este rejectata. Pentru detalii vezi \citep{wilcoxon45,siegel56}.
    \Section{Rezultate}
    \subsection{Populația}
    Un număr de 50 de pacienți au fost observați. Dintre aceștia 62\% (N=31) sunt de sex feminin iar 38\% (N=19) sunt de sex masculin (proporția sexelor în grupa populației urbane cu vârste cuprinse intre 27 și 83 ani la nivel național conform \citep{insee2011} este de 47\% M și 53\% F).
    Vârsta pacienților de sex feminin este distribuita normal în jurul mediei de 50 de ani și 7 luni ($\sigma=14.3,min=27,max=77$) iar cea a pacienților de sex masculin este o combinație de distribuții normale centrate în jurul mediilor de 46 respectiv 75 ani ($\sigma_{1}=12.3 , \sigma_{2}=9.2,min=30,max=83$).
    %asTheEconomist(densityplot(~Varsta,data=IU,groups=Tip,auto.key=T,plot.points=F,main='',panel=function(...){
    %	panel.densityplot(...); #deseneaza graficul initial
    %	panel.xyplot(0:100,0.6*dnorm(0:110,46,12.3)+0.38*dnorm(0:110,75,9.2)) #deseneaza un kernel 
    %}))
    Pentru a evalua reprezentativitatea eșantionului relativ la distribuția vârstelor în cadrul populației din Romania am apelat la datele oficiale din \citep{insee2011} care detaliază numărul de cetățeni romani pe sexe și categorie urban/rural pentru fiecare vârstă la data de 1 iulie 2010. Analiza statistica s-a efectuat folosind testul Wilcoxon iar concluzia este ca atât eșantionul de sex feminin ($p=0.005193$) cât și cel de sex masculin($p<2.2*10^{-16}$) corespund cu distribuția generala în populația urbana a României.
    \begin{figure}[H]
	\centering
	\includegraphics[width=0.8\linewidth]{incoVarstaSex}
	%dp1<-asTheEconomist(densityplot(~Varsta,data=IU,groups=Tip,auto.key=T,plot.points=F,main='',ylab=''))
	%dp2<-bwplot(Varsta~Tip,data=IU,ylab='',par.settings=theEconomist.theme())
	%dp1$yscale.components.old<-dp1$yscale.components
	%dp1$yscale.components<-function(...) { ret<-dp1$yscale.components.old(...); ret$left$labels$labels=NULL; ret}
	%print(dp1,split = c(1, 1, 1, 2))
	%print(dp2,split = c(1, 2, 1, 2),newpage=FALSE)
	\caption{Distribuția sexelor participanților la studiu}
	\label{fig:Distributia sexelor participantilor la studiu}
    \end{figure}
  Din punct de vedere al greutății am evaluat indicatorul \ac{BMI} conform cu pragurile recomandate de \citep{whobmi06}. Astfel, pentru sexul feminin avem 13 persoane cu greutate normala ($BMI<25.0$, NOR), 16 supraponderale ($25.0 \geq BMI <30.0$, OVR) și 2 obeze($BMI \geq 30.0$, OBE). Pentru sexul masculin avem 3 persoane cu greutate normala, 12 supraponderale și 4 obeze.   
  \begin{table}[H]
  \centering
  \begin{tabular}{ |l|l|l|l| }
  \hline
  Sex & NOR & OVR & OBE \\ \hline
  F & 13 & 16 & 2 \\ \hline
  M & 3 &  12 & 4 \\ \hline
  \end{tabular}
  \caption{Numărul de persoane din fiecare categorie BMI pe sexe}
  \label{tab:BMIgSex}
\end{table}

    \begin{figure}[H]
	\centering
	\includegraphics[width=0.8\linewidth]{incobmiDens}
	\caption{Distribuția BMI pe sexe, Zona galbena marchează persoanele supraponderale și cea roșie pe cele obeze}
	\label{fig:incobmiDens}
    \end{figure}
  Distribuția BMI pe grupa de vârstă și pe sexe am evaluato la nivel național conform \citep{EHIS09} care oferă informații detaliate despre incidenta problemelor de nutriție în rândul tarilor membre ale Uniunii Europene. Din cauza eșantionului foarte mic, nu se poate trage concluzia ca populația studiata este diferita de un eșantion aleator la nivel național dar examinând graficul din figura alăturata se poate observa (cu excepția unor situații particulare - de exemplu toate persoanele de sex masculin din grupa de vârstă 25-44 ani sunt supraponderale sau obeze) ca valorile procentelor urmăresc distribuția naționala.
  \begin{figure}[H]
	\centering
	\includegraphics[width=0.8\linewidth]{incoBMIvsEHIS-OOB}
	\caption{Distribuția procentului de persoane obeze în populația studiata (EIU) și în populația generala (EHIS) }
	\label{fig:incoBMIvsEHIS-OOB}
  \end{figure}
  
    \closeopenmulticols
    \begin{table}[H]
     \centering
    \begin{tabular}{ |l|l|l|l|l| }
     \hline
     Grupa de vârstă & Sex & Categorie BMI & Număr persoane & Procent \\ \hline
    25-44 & F & NOR & 7  & 63.6 \\ \hline
    25-44 & F & OVR & 3  & 27.3 \\ \hline
    25-44 & F & OBE & 1  &  9.1 \\ \hline
    25-44 & M & NOR & 0  &  0.0 \\ \hline
    25-44 & M & OVR & 4  & 80.0 \\ \hline
    25-44 & M & OBE & 1  & 20.0 \\ \hline
    45-64 & F & NOR & 3  & 21.4 \\ \hline
    45-64 & F & OVR & 10 & 71.4 \\ \hline
    45-64 & F & OBE & 1  &  7.1 \\ \hline
    45-64 & M & NOR & 1  & 20.0 \\ \hline
    45-64 & M & OVR & 3  & 60.0 \\ \hline
    45-64 & M & OBE & 1  & 20.0 \\ \hline
    65-74 & F & NOR & 3  & 50.0 \\ \hline
    65-74 & F & OVR & 3  & 50.0 \\ \hline
    65-74 & F & OBE & 0  &  0.0 \\ \hline
    65-74 & M & NOR & 2  & 22.2 \\ \hline
    65-74 & M & OVR & 5  & 55.6 \\ \hline
    65-74 & M & OBE & 2  & 22.2 \\ \hline
    \end{tabular}
  \caption{Numărul de persoane și procentul din totalul de persoane dintr-o grupa de vârstă din fiecare categorie BMI pe sexe și pe grupa de vârstă}
  \label{tab:bmigCounts}
 \end{table}
  \begin{multicols}{2}
 Studiul a înregistrat și  date referitor la co-morbiditatea 

        \closeopenmulticols
  \bibliographystyle{plainnat}
  \bibliography{incoStudy}
  
  \section*{Glosar}
  \begin{acronym}[LUTS]
    \acro{IU}{Incontinența Urinară}
    \acro{IUI}{Incontinența Urinară prin Imperiozitate}
    \acro{IUE}{Incontinența Urinară de Efort}
    \acro{IUM}{Incontinența Urinară Mixtă}
    \acro{LUTS}{Lower Urinary Tract Symptoms}
    \acro{SEP}{Stimulare Electrica Periferica}
    \acro{BMI}{Body-Mass Index}
  \end{acronym}
 
 \listoffigures
 \listoftables
   
\end{document}
